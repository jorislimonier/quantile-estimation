\documentclass{article}

\usepackage{graphicx}
\usepackage{float}
\usepackage{caption}
\usepackage{subcaption}
\usepackage{multirow}
\usepackage{hyperref}
\hypersetup{
    colorlinks  = true,
    linkcolor   = blue,
    filecolor   = blue,      
    urlcolor    = blue,
}

\title{Student job report: Bootstrap high quantiles estimation.}
\author{Joris LIMONIER}
\date{February - May 2021}

\begin{document}
\maketitle

\newpage
\tableofcontents
\listoffigures
\newpage

\section{Introduction}
In this project we want to estimate the confidence we have when computing various quantiles. This task may be fairly straightforward when working with known distributions or arbitrary amounts of data but this seldom occurs in engineering or other real world applications. For this reason, here we try to accomplish that task on an unknown distribution and given limited amounts of data.

We define the n-th quantile as follows:
\begin{equation}
    q_n := 1 - 10^{-n}
\end{equation}
which gives $q_1 = 0.9$, $q_2 = 0.99$ ...etc. Where, simply speaking, $q_n$ represents ``0" followed by $n$ nines. For our applications, we are mostly interested in $q_3, q_4$ and $q_5$.

The data we work with is presented in figure \ref{fig: initial data} and plotted in the histogram in figure \ref{fig: visual representation of the initial data}.

\begin{figure}[H]
    \centering
    \begin{subfigure}{.25\textwidth}
        \centering
        \includegraphics[width=\textwidth]{images/excel_data.png}
        \caption{Initial data}
        \label{fig: initial data}
    \end{subfigure}
    \hfill
    \begin{subfigure}{.74\textwidth}
        \centering
        \includegraphics[width=\textwidth]{images/plot_excel_data.png}
        \caption{Visual representation of the initial data}
        \label{fig: visual representation of the initial data}
    \end{subfigure}
    \caption{A look at the initial data}
\end{figure}

\section{Bootstrap}
\label{section: bootstrap}
We use the bootstrap to estimate the confidence in the computed quantiles, even with small to moderate sample size.

Let \(n\) be the size of the data at our disposal and let \(q_k\) be the quantile we want to compute. Here is how we proceed:
\begin{enumerate}
    \item Do the following \(n\) times.
          \begin{enumerate}
              \item Draw a random sample of size \(n\) with replacement from the initial data.
              \item Compute \(q_k\) on the sample which has just been drawn. \label{bootstrap repeated step}
          \end{enumerate}
    \item Compute the mean over the set of values resulting from step \ref{bootstrap repeated step}.
    \item Compute the confidence intervals (details in section \ref{section: confidence intervals on the bootstrap})
\end{enumerate}

The plots in figure \ref{fig: estimation of quantiles} show the evolution of our estimate for the value of the quantiles as we go through runs of the bootstraps. The grey areas represent the $95\%$ confidence intervals during that evolution. We will see how to get the confidence intervals from our histogram in section \ref{section: confidence intervals on the bootstrap}.

\begin{figure}
    \centering
    \begin{subfigure}{.84\textwidth}
        \includegraphics[width=\textwidth]{images/estimation_q3.png}
        \caption{Estimation of $q_3$}
    \end{subfigure}
    \hfill
    \begin{subfigure}{.84\textwidth}
        \includegraphics[width=\textwidth]{images/estimation_q4.png}
        \caption{Estimation of $q_4$}
    \end{subfigure}
    \hfill
    \begin{subfigure}{.84\textwidth}
        \includegraphics[width=\textwidth]{images/estimation_q5.png}
        \caption{Estimation of $q_5$}
    \end{subfigure}
    \caption{Estimation of the quantiles over bootstrap runs}
    \label{fig: estimation of quantiles}
\end{figure}

\section{Confidence intervals on the bootstrap}
\label{section: confidence intervals on the bootstrap}

Let us assume that we want to compute the $95\%$ confidence interval. \\
First, let us note the following:
\begin{equation}
    \label{eq: compute 0.95 CI bounds}
    \frac{1 - 0.95}{2} = 0.025 \qquad \mbox{and} \qquad 1 - \frac{(1 - 0.95)}{2} = 0.975
\end{equation}

Now we compute the $95\%$ confidence interval. First, we sort the set of all the $q$'s that have been computed for each individual sample. Then we take the 0.025-th and 0.975-th quantile respectively as our lower-bound and upper bound for the confidence intervals. \\
More generally, for a $\gamma$-confidence interval (instead of $95\%$), one has that the lower and upper bounds of the confidence interval are respectively
\[
    \gamma_{lo} := \frac{1 - \gamma}{2} \qquad \mbox{and} \qquad \gamma_{up} := 1 - \frac{(1 - \gamma)}{2}
\]


\textbf{According to our data}, it seems that there is close to no change in the estimation of the quantiles after $1000$ repetitions of the bootstrap. The variations are small after 500 runs already but for safety purposes we consider that we have our final guess after 1000 runs. As for the confidence intervals, only in some edge cases do we have changes past the 1000 mark. \\
\textit{NB: Our estimate of 1000 runs is based on empirical evidence. It is not a theoretical result, however, we believe that it is suitable for engineering purposes.}

The fact that our estimate seems to be stable after \(1000\) runs does not matter for the number \(1000\) itself. However, it matters because we seem to ``converge" to some value and reach a final value.







\end{document}